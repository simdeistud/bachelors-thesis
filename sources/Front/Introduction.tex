\chapter{Introduction}
% Fill with what you want to include in the introduction
\epigraph{"Simplicity is a great virtue but it requires hard work to achieve it and education to appreciate it. And to make matters worse: complexity sells better."}{\textit{Edsger W. Dijkstra}}

\section*{What is an ISA?}

A computer is a device which is capable of acquiring data, performing calculations upon it, and making the results available for use at a later date.
It is clear from this definition, that when deciding how to design and build a computer one must at least take into consideration the way data is 
stored and organized (the memory) and the mechanisms through which the computer is able to manipulate said data (the processor).
Computers are an abstract concept and do not impose a certain technological choice to their physical realization. Nonetheless, the vast majority of computers nowadays
are built through the assembly of digital components and thus natively speak the language of the binary number system.
As such, just like when using a mechanical device an operator needs to interact with the physical parts of the system,  operating a computer at this level,
would require the user to manually insert ones and zeros into the right places for it to perform its calculations.
It is clear that such an operation would require an intimate knowledge of the physical implementation of the computer, and even minimal
changes to its digital circuitry might jeopardize the correctness of any sequences of bits written for an earlier model.

Early on in the history of computers it was understood that an additional layer of abstraction was needed in order to separate the hardware from the software
and give more freedom both to the circuit designers and the programmers. This layer of abstraction is called an Instruction Set Architecture,
which from now on will be called ISA for short. An ISA provides a logical specification of how a computer manages its memory and what the instructions that it's
capable of performing are. This forms the layer through which all software must interface with in order to interact to the hardware.

\section*{What is the LEGv8 ISA?}

The ISA focus of this thesis is the LEGv8 ISA. This ISA was created by David A. Patterson and John L. Hennessy to serve as a teaching
tool in their book \emph{Computer Organization and Design (ARM Edition)}. As the title suggests, the book is actually about the ARMv8 ISA, whose first
version was originally released in 1983 by Acorn Computers and which now is designed by Arm Holdings plc. The authors however introduced a few
changes and simplifications to the ARMv8 ISA to make it friendlier to students and emphasize certain design concepts. As such, this ISA is used
in the sections of the book dedicated to the design of a model processor and its programming, and its these sections upon which the LEGv8 simulator
subject of this thesis is based.

\section*{Overview of the LEGv8 ISA}
