\chapter{Introduction}
% Fill with what you want to include in the introduction

\epigraph{``The purpose of abstracting is not to be vague, but to create a new semantic level in which one can be absolutely precise.''}{\textit{Edsger W. Dijkstra} -- The Humble Programmer (1972)}

\paragraph{}
The work presented in this thesis concerns the restoration and development of a Java-based LEGv8 simulator.

\section*{Some context}
\paragraph{}
LEGv8 is an Instruction Set Architecture (which from now on will be referred to as \emph{ISA}) designed by D.A. Patterson and J.L. Hennessy to serve as a pedagogical tool in their textbook \emph{Computer Organization and Design
ARM Edition} \cite{patterson2016computer}.
An ISA is a set of specifications that provides a logical description of a processor-based system. It offers a layer of abstraction that both the hardware and the software need to conform to in order to ensure proper interoperability, meaning that code written for a certain ISA is sure to work on any processor designed following the same specification, and vice versa.
As the title of the book suggests, the authors use the ARM ISA (more specifically, the v8 version) as a real life showcase of the modern approaches to the design of computer systems. However, in some of the chapters, a simpler architecture is needed to better present some of the concepts, and for this reason the authors identified a simpler and more coherent subset of the ARMv8 ISA which they named LEGv8. 
Being a fictional ISA created to serve the pedagogical purposes of a single textbook, there are currently no commercially available hardware implementations of the architecture defined by LEGv8, meaning that any code written for it can only be executed by the means of software simulation.

\paragraph{}
Being a relatively popular textbook in undergraduate Computer Architeture courses, an ecosystem of simulators has developed over the years, mostly from the enterprise of students and professors, either for personal interest or because of course requirements. These simulators can be divided in two categories: hardware simulators and software simulators. The former concern themselves with simulating the inner workings of a hypothetical physical implementation of the LEGv8 ISA, while the latter are more interested in simulating the logical execution of the code from the point of view of an external user. Regardless of the type, none of them achieve a complete simulation of the entire LEGv8 ISA for one reason or another, especially when dealing with floating point operations. This might not be a issue when wanting to execute, test, or showcase code limited to only a few of the available instructions, as would be the case in a university course with well defined requirements, but problems arise when a student or educator might want to do so using the full spectrum of operations specified by LEGv8. For this reason, a simulator capable of achieving a complete and correct coverage of the entire ISA would be a welcomed addition in the current LEGv8 software landscape.

\paragraph{}
Among the software simulators available online, one is being developed and published by Arm itself \cite{legv8simARMrepo}, and is the subject of this thesis' work.
Arm's LEGv8 simulator is almost entirely written in Java and uses the GWT framework \cite{gwtweb} to deploy the software in the form of a web application. Its codebase is well structured and its use of a platform-agnostic and high level language allows for easy development and extension. Furthermore, the use of GWT gives it the ability to be run on any device running a modern-ish version of any HTML5-compatible web browser. Due to GWT's integration with the Eclipse IDE for Java, this development environment was chosen for the project. These design choices, however, present some criticalities which will be discussed later.
Other issues arise concerning the execution of the simulator, with bugs impeding the implementation of conditional statements, loops, the correct execution of user-defined functions. It is clear then, that a simulator only capable of executing branchless code without support for code reuse, is not adequate for the simulation of the vast majority of computer code.
\paragraph{}
As mentioned in the previously, none of the publicly available simulators cover the entire LEGv8 ISA, and Arm's is no exception. This, combined with the presence of the aforementioned simulation-breaking bugs, might discourage its usage and even question its utility. Nonetheless, because of its popularity and the nature of its publisher, it has been deemed the highest impact project when deciding which one to work on for this thesis. 

\section*{Presentation of the thesis' work}
\paragraph{}
The work presented in this thesis has tackled different parts both internal and external to the simulator, and can be divided into 5 logical phases, not necessarily in chronological order.

\paragraph{0. Study of the simulator.}
The initial part of the work has consisted in analyzing the structure and inner workings of the simulator, in order to identify the developers' intentions with its design and the source of its critical bugs. An important part of this work has been understanding the codebase's undocumented build process and dependencies, documenting them through a comprehensive tutorial, and making it publicly available to allow for future development and collaboration.
\subparagraph{}
An overview of the LEGv8 ISA, a survey of the available simulators, and an overview of Arm's codebase together with its set-up shortcomings and the work done fixing them will be the subject of Chapter \ref{chap:chap1}.

\paragraph{1. Modernization of the development environment.}
A few of the design choices made by the original developers have had some unfortunate consequences on the current state of the project. First among them has been the lack of documentation concerning the build process for the simulator, which has made it impossible to effectively develop fixes and enhancements to the codebase, deploy them, and share them. Secondly, the choice of GWT as a framework upon which to base the project, has caused the developers to depend on an outdated version of the Eclipse IDE for development. Furthermore, there have been some changes to newer versions of GWT that break the original Eclipse's workflow.
\subparagraph{}
The work done in this phase has been to update GWT to its latest iteration, and to adapt the codebase to accomodate the new build environment. This was achieved by integrating Maven \cite{mavenweb} into the project, thus bringing it to a modern standard of build automation and dependency management. Thanks to this work, the simulator's development environment has been decoupled from Eclipse and can now be configured in a few simple steps, both from the terminal or via any Maven-compatible Java IDE. This allows for an almost automatic configuration process, and opens the doors to better development and collaboration techniques, thanks to the ability to utilize different and more featureful IDEs, new language features, and take advantage of the capabilities provided by Maven such as code testing and plugins.
\subparagraph{}
A more extensive overview of the project's structure, its dependencies, configuration, and the steps taken to modernize it will be included in Chapter \ref{chap:chap2}.

\paragraph{2. Critical bug fixes.}
As mentioned before, the simulator contained some breaking bugs that influenced the expected flow of code execution.
The first bug consisted in the incorrect evaluation of comparisons between integers. Since integer comparisons are how conditional statements are implemented, this caused branches to not be taken and loops to not function properly.
The second bug was instead the result of a problem in setting the correct return address when calling a user defined function. This meant that, once the functions was done executing, instead of returning to the main body of the program where the function was called, the function was executed again, causing an infinite loop.
\subparagraph{}
The nature of these bugs, and the steps taken to fix them, will be discussed in Chapter \ref{chap:chap3}.

\paragraph{3. Completing the integer arithmetic and integrating floating point support.}
From the survey of available simulators, Arm's didn't distinguish itself for its coverage of the LEGv8 instruction set. This is both the case for the integer instructions (having implemented only a subset) and for the floating point ones (none of which have been implemented).
Concerning the addition of the missing integer instructions, the structure of the codebase allowed for an easy augmentation, although some limitation of the Java language made necessary the use of alternative solutions. None of these instructions fundamentally changed the inner workings of the simulator, thus all that was needed was to go along with the already implemented design.
Adding support for floating-point operations, on the other hand, introduced new elements to the design of the simulator, since it required separate places in which to store the values being operated with, and didn't allow the intermixing of integer and floating point values and instructions. In order to implement these features, some previously untouched parts of the simulator have been extended with the necessary logic.

\paragraph{4. Updating the simulator's UI.}
All the changes and additions mentioned in the previous paragraph needed to be presented to the user through an update to the original UI.
An initial addition, independent from all the changes already mentioned, was the introduction of the visualization of the main memory of the simulated computer. This allows the user to keep at eye on the stack in real time, both to learn how it works, and to check for errors in the code.
Secondly, all the new floating-point capabilities required the addition of new elements in the UI to properly visualize the now expanded state of the simulator.
Lastly, a general reorganization of the UI elements was performed to make the simulator fit better on an average computer monitor or projector, and increase the density of the presented information.
\subparagraph{}
A complete overview and explanation of all the introduced changes and additions, both to the internal logic of the simulator and its UI, will be provided in Chapter \ref{chap:chap4}.

\paragraph{}
The work presented in this thesis only tackled a subset of the problems and shortcomings of the simulator, as they are numerous and would require a great amount of additional work, as many of them are intrinsic to the design choices of the original developers. The concluding chapter will include a final overview of the work done and a discussion of the possible further developments to the simulator and how they might be achieved.