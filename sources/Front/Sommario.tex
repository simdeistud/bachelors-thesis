\chapter{Sommario}\label{chap:somm}
\paragraph{}
LEGv8 è un'ISA ispirata ad ARMv8 e sviluppata da D.A. Patterson e J.L. Hennessy per il loro libro di testo sulle Architetture dei Sistemi Digitali \cite{patterson2016computer}. Vista la sua natura puramente didattica, l'esecuzione del suo codice nativo è relegata alla simulazione software.
Tra i simulatori disponibili, è stato scelto come oggetto di questa tesi un simulatore basato su Java \cite{legv8simARMrepo} pubblicato da Arm \cite{armweb}. Esso si distingue per il suo codice ben strutturato e la sua distribuzione come applicativo web. I suoi principali svantaggi sono la mancanza di documentazione su come configurare e compilare il simulatore, librerie obsolete, un'implementazione incompleta dell'ISA, e una serie di bug critici che impediscono il normale funzionamento della simulazione.
\paragraph{}
Il lavoro presentato in questa tesi può essere diviso in cinque parti:
\begin{enumerate}
    \item Comprensione e documentazione dello stato attuale della base di codice e i modi per compilarla e distribuirla come previsto dagli sviluppatori.
    \item Modernizzazione delle librerie utilizzate, e integrazione con un moderno sistema di build automation \cite{mavenweb} per rendere lo sviluppo e la distribuzione del simulatore il più semplici possibile, e renderlo indipendente dagli strumenti con cui è stato originariamente sviluppato.
    \item Correzione dei bug critici che impedivano il normale funzionamento del simulatore, ovvero l'incapacità di eseguire confronti tra interi e chiamate a subroutine.
    \item Implementazione dell'intera ISA LEGv8 a ciclo singolo, aggiungendo le mancanti istruzioni intere e aggiungendo il supporto per operazioni a virgola mobile.
    \item Aggiornamento  dell'interfaccia del simulatore per fornire una visualizzazione in tempo reale dello stack, delle nuove istruzioni intere e a virgola mobile, e dei registri a virgola mobile. Inoltre, una riorganizzazione del layout per presentare meglio lo stato del simulatore.
\end{enumerate}


