\chapter{Guide for the new Maven-based set-up}\label{chap:appB}
\paragraph{}
In this brief appendix, we will provide the tools to build the updated version of AceGWT and some tutorials on how to use the updated build system on a variety of Java IDEs supporting Maven.
\section{Maintaining AceGWT}
\paragraph{}
As was touched upon in Chapter \ref{chap:chap2}, in order to update AceGWT to the latest GWT version, some changes had to be made. Unfortunately, they are currently not present in the official GitHub repository \cite{web:acegwtgit}, and thus a custom version of the library needs to be built and used. Some additional documentation about the project is available on the original repository's wiki.
\paragraph{Creating parent pom.xml}
First of all, to aid the build process, it is useful to create a pom.xml file in the root of the project. This is reported in Listing \ref{lst:acegwtparentpom}.
\begin{lstlisting}[float, language=XML, label={lst:acegwtparentpom}, caption={The new parent pom.xml}]
<?xml version="1.0" encoding="UTF-8"?>
<project xmlns="http://maven.apache.org/POM/4.0.0"
         xmlns:xsi="http://www.w3.org/2001/XMLSchema-instance"
         xsi:schemaLocation="http://maven.apache.org/POM/4.0.0 http://maven.apache.org/xsd/maven-4.0.0.xsd">
  <modelVersion>4.0.0</modelVersion>
  <groupId>edu.ycp.cs.dh</groupId>
  <artifactId>acegwt-2.11.0</artifactId>
  <version>1.3.3</version>
  <packaging>pom</packaging>
  <name>AceGWT-2.11.0</name>
  <description>
    Ace is an embeddable code editor written in JavaScript. It matches the features and performance of native editors such as Sublime, Vim and TextMate. It can be easily embedded in any web page and JavaScript application. Ace is maintained as the primary editor for Cloud9 IDE and is the successor of the Mozilla Skywriter (Bespin) project. AceGWT is an integration of Ace into GWT.
  </description>
  <modules>
    <module>AceGWT</module>
    <!-- <module>AceGWTDemo</module> -->
  </modules>
</project>
\end{lstlisting}
\paragraph{Updating children pom.xml files}
Due to the new ownership of the GWT project, new artifact names are needed to include GWT 2.11. Listing \ref{lst:acegwtchildrenpom} shows the changes required for both the AceGWT and AceGWTDemo modules.
\begin{lstlisting}[float, language=XML, label={lst:acegwtchildrenpom}, caption={The changes to the children's pom.xml}]
...
<parent>
  <groupId>edu.ycp.cs.dh</groupId>
  <artifactId>acegwt-2.11.0</artifactId>
  <version>1.3.3</version>
</parent>
...
<gwt.version>2.11.0</gwt.version>
...
<dependencyManagement>
  <dependencies>
    <dependency>
      <groupId>org.gwtproject</groupId>
        <artifactId>gwt</artifactId>
        <version>${gwt.version}</version>
        <type>pom</type>
...
<dependencies>
  <dependency>
    <groupId>org.gwtproject</groupId>
    <artifactId>gwt-user</artifactId>
  </dependency>
  <dependency>
    <groupId>org.gwtproject</groupId>
    <artifactId>gwt-dev</artifactId>
  </dependency>
  <dependency>
    <artifactId>gwt-codeserver</artifactId>
    <scope>provided</scope>
  </dependency>
  <dependency>
    <groupId>org.gwtproject</groupId>
    <artifactId>gwt-servlet</artifactId>
    <scope>runtime</scope>
  </dependency>
...
\end{lstlisting}
\paragraph{Updating module.gwt.xml}
The \verb|module.gwt.xml| file, in both children modules, needs only an update to the link of the DTD, as shown in Listing \ref{lst:acegwtdtd}.
\begin{lstlisting}[float, language=XML, label={lst:acegwtdtd}, caption={Updated DTD version}]
<!DOCTYPE module PUBLIC "-//Google Inc.//DTD Google Web Toolkit 2.11.0//EN"
        "http://gwtproject.org/doctype/2.11.0/gwt-module.dtd">
\end{lstlisting}
\paragraph{Building the library}
Now the library has to be built, in our case,  into a .jar file. This can be achieved by running\\ \verb|mvn clean gwt:generate-module compile gwt:package-lib| into the \verb|AceGWT| folder inside the project. This will create a folder called \verb|target|. The contents of this folder are the ones that need to be copied in the local Maven repo folder discussed in Chapter \ref{chap:chap2}.
\section{Set-up tutorials for various Java IDEs}
\paragraph{}
We will present written descriptions of the steps needed to set-up the project in the most popular Java IDEs that support Maven. It is possible to develop the simulator even in environments without Maven support, such as a normal text editor, by editing the code and running Maven via the terminal as showcased in the end of Chapter \ref{chap:chap2}.
\subsection{IntelliJ IDEA}
\paragraph{}
Choose the \verb|Get from VCS| option and copy paste this repository's git link. The Maven project will be cloned into your workspace and the dependencies will be automatically downloaded and configured. After making your changes you can build and package the simulator by going to the Maven panel on the right, navigate to \verb|Graphical-Micro-Architecture-Simulator| $\longrightarrow$ \verb|LEGv8_Simulator| $\longrightarrow$ \verb|Lifecycle| $\longrightarrow$ \verb|package|. The folder containing the simulator will appear under \verb|LEGv8_Simulator/target/|.
IntelliJ Ultimate offers a GWT plugin that warns the programmer when using methods, syntax and classes not implemented by GWT and can generate compile reports. Be warned that in some cases it might show bogous errors (such as missing css directives), but this shouldn't affect the compilation which is done with GWT. If you use IntelliJ without this plugin the errors will disappear but you will not have access to said features.
\subsection{Eclipse}
\paragraph{}
Choose \verb|Import projects...|, select\newline \verb|Git/Projects from Git (with smart import)| $\longrightarrow$ \verb|Clone URI| and copy paste this repository's git link. Keep pressing \verb|Next| until it has finished the procedure. It is recommended to go to \verb|Window| $\longrightarrow$ \verb|Preferences| $\longrightarrow$ \verb|XML (Wild Web Developer)| and enable the download of external resources. Unlike IntelliJ, Eclipse doesn't show directly the package action but has to be added manually by right clicking on the \verb|LEGv8_Simulator| folder and going to \verb|Run As| $\longrightarrow$ \verb|Run Configurations...| and then double click on \verb|Maven Build|. This will create a new configuration for you to edit: give it the name package, select \verb|Workspace...| $\longrightarrow$ \verb|LEGv8_Simulator| as the \verb|Base Directory| and write \verb|package| inside the \verb|Goals| text box. Now you can \verb|Apply| and run it. To run it again just go to the \verb|Run As| menu as before and it should have been added there. The folder containing the simulator will appear under \verb|LEGv8_Simulator/target/| (press F5 in Eclipse to refresh the folders).
Eclipse offers a GWT plugin (which has installation problems with Eclipse versions newer than the 2023-09) that makes compilation easier but, unlike the Maven package action, deploys the compiled sources into a \verb|/war| folder, and doesn't automatically copy-paste the web resources needed to launch the web page. That has to be done manually. This plugin uses the older build method and is not recommended.
\subsection{Apache NetBeans}
\paragraph{}
Clone the repository to a location of your choosing. In NetBeans go to \verb|File| $\longrightarrow$ \verb|Open Project...| and select the cloned repository folder. To access the \verb|LEGv8_Simulator| files in the IDE go to\newline \verb|Graphical-Micro-Architecture-Simulator| $\longrightarrow$ \verb|Modules| and double click on \verb|LEGv8_Simulator|. This will open the module in the project browser. In order to build and package the simulator, right click on \verb|LEGv8_Simulator| $\longrightarrow$ \verb|Run Maven| $\longrightarrow$ \verb|Goals...| and write package into the \verb|Goals| text field and press \verb|OK|. The folder containing the simulator will appear under\newline \verb|LEGv8_Simulator/target/|.