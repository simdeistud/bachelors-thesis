              %******************************************%
              %                                          %
              %         Standard thesis template         %
              %            by Stefano Bianchi            %
              %       translation by Simone Deiana       %
              %                                          %
              %          version: 27 June 2024           %
              %                                          %
              %******************************************%
       
% If one wants to print on both sides, substitute 'oneside' for 'twoside'
% in the parameters of \documentclass
\documentclass[a4paper,11pt,oneside,openany]{book}

\usepackage[english]{babel}

\usepackage[utf8]{inputenc} % Enables the use of accented Italian characters
\usepackage[T1]{fontenc} % Enables the use of special characters usually not present in English alphabets

\usepackage{fancyhdr} % Allows for the rendering of the cover

\usepackage{graphicx} % Allows for the inclusion of figures in the document
\usepackage{epigraph}
\usepackage{enumitem}
\usepackage{xcolor}
\usepackage{lscape}
\usepackage{wrapfig}
\usepackage{listings}
\usepackage{float}
\definecolor{mygreen}{rgb}{0,0.6,0}
\definecolor{mygray}{rgb}{0.5,0.5,0.5}
\definecolor{mymauve}{rgb}{0.58,0,0.82}
\definecolor{light-gray}{gray}{0.80}

\lstset{
	backgroundcolor=\color{white},   % Choose the background color
	basicstyle=\footnotesize\ttfamily, % Set the font style
	breakatwhitespace=false,         % Break lines only at white space
	breaklines=true,                 % Break long lines
	captionpos=b,                    % Position of caption (b: bottom)
	commentstyle=\color{gray},    % Custom color for comments
	keywordstyle=\color{blue},       % Custom color for keywords
	numberstyle=\tiny\color{mygray}, % Custom color for line numbers
	stringstyle=\color{mygreen},    % Custom color for strings
	frame=single,                    % Add a frame around the code
	keepspaces=true,                 % Keep spaces in text
	language=Java,                   % Specify the programming language
	numbers=left,                    % Show line numbers on the left
	showspaces=false,                % Don't show spaces
	showstringspaces=false,          % Don't show spaces in strings
	showtabs=false,                  % Don't show tabs
    moredelim=[is][\color{red}]{|}{|},
	tabsize=2                        % Set tab size
}

\lstdefinelanguage{XML}
{
  morestring=[b]",
  stringstyle=\color{mygreen},
  identifierstyle=\color{black},
  keywordstyle=\color{blue},
  morekeywords={xmlns,version,type,modules,description,name,packaging,artifactId,groupId,modelVersion,parent,dependencyManagement,dependencies,dependency,scope,gwt}% list your attributes here
}

\usepackage{subfigure} % Allows for the inclusion of subfigures in the document

\usepackage{tabularx} % Adds functionalities to tables
\usepackage{longtable} % Allows to write tables on multiple pages

\usepackage[titletoc]{appendix} % Allows to put appendix chapters in the index

\usepackage{float} % Needed to include floating images

\usepackage{fancyvrb} % Allows for the use of Verbatim windows (i.e. unformatted text) inside a frame

\usepackage{amsmath} % Adds functionalities to formulas
\numberwithin{equation}{section}

\usepackage{hyperref} % Index of the topics with clickable link, for a more comfortable document navigation

\usepackage{lipsum} % Allows for the insertion of filler text inside the thesis; useless when writing. Delete this command and all the \lorem commands inside the text.

\usepackage{easyReview}

% To add content, refer to the relative files marked by parenthesis in the parameters of \input

%*******************************************************
% Settings of the single page and the cover
%*******************************************************
\hypersetup{ % Visualization of clickable links
    colorlinks,
    citecolor=green,
    filecolor=black,
    linkcolor=red,
    urlcolor=blue
}

% Sets the pages' header
\fancyhf{}
\pagestyle{fancy}
\lhead{}
\chead{\leftmark}
\rhead{}
\lfoot{}
\cfoot{}
\rfoot{\thepage}

%*******************************************************
% Cover
%*******************************************************
% Configures the cover
% Do not modify this part
\usepackage[some]{background}
\SetBgScale{1}
\SetBgContents{
\includegraphics[scale=1]{img/units_logo.pdf}}
\SetBgColor{gray}
\SetBgAngle{0}
\SetBgOpacity{0.07}
% Do not modify this part

% Title of the thesis and name of the author
\title{Reverse engineering per la decodifica di filmati di videosorveglianza in formato proprietario}
\author{Name Surname}
\begin{document}
\pagenumbering{roman}
\begin{titlepage}
    \begin{center}
    % Insert the background here (front page)
    \BgThispage
    {\LARGE {\bfseries UNIVERSITY OF TRIESTE \\}}
    \vspace{.5cm}
    {\Large {\bfseries Department of Engineering and Architecture \\}}
    \vspace{1cm}
    \includegraphics[width=6cm,height=6cm]{img/units_logo.pdf}\\[1.5cm]

    {\LARGE
        Master's degree in Computer Engineering \\
    }
    \vspace{1cm}
    {\LARGE 
        {\bfseries Reverse engineering per la decodifica di filmati di videosorveglianza in formato proprietario}
    }
    \vspace{1cm}

    % Remember to update the date before printing the document :)
    {\large \today \\
    }

    % Table to report the names of the graduating student, supervisor and co-supervisor (if present)
    \vfill
    \begin{table}[h]
        {\large
            \begin{tabular}{c c c c r c c | c c l}
                & & & & Graduating student & & & & & Supervisor \\
                & & & & \bfseries Name Surname & & & & & \bfseries Prof. Sergio Carrato \\ % Modify the name of the graduating student
                & & & & & & & & & \\
                & & & & & & & & & Co-supervisor \\ % Delete these two lines,
                & & & & & & & & & \bfseries Tit. Name Surname \\ % if there is no co-supervisor.
            \end{tabular}
        }
    \end{table}
    Academic Year 2013/2014
    \end{center}
\end{titlepage}


%*******************************************************
% Dedications
%*******************************************************
% Do not modify this part
\phantomsection
\thispagestyle{empty}
\vspace*{3cm}
% Do not modify this part

\begin{center}
Così tra questa \\
immensità s'annega il pensier mio: \\
e il naufragar m'è dolce in questo mare. \\
--- Giacomo Leopardi -- da \emph{L'infinito} ---
\end{center}

\medskip

\begin{center}
\hfill \emph{Dedication 1} \\
\hfill \emph{Dedication 2} \\
\hfill \emph{...} \\
~ \\
\hfill \emph{Other sentences} \\
\hfill \emph{to dedicate} \\
\hfill \emph{to loved ones} \\
\end{center}

\frontmatter

%*******************************************************
% Summary
%*******************************************************
\chapter{Summary}

\lipsum[1]
\newpage

%*******************************************************
% Italian summary
%*******************************************************
\chapter{Sommario}

\lipsum[1]
\newpage

%*******************************************************
% Index
%*******************************************************
\tableofcontents
\newpage

%*******************************************************
% Introduction
%*******************************************************
% Do not modify this part
\renewcommand{\chaptermark}[1]{\markboth{\MakeUppercase{\ #1}}{}}
% Do not modify this part

\chapter{Introduction}
% Fill with what you want to include in the introduction

\epigraph{``The purpose of abstracting is not to be vague, but to create a new semantic level in which one can be absolutely precise.''}{\textit{Edsger W. Dijkstra} -- The Humble Programmer (1972)}

\paragraph{}
The work presented in this thesis concerns the restoration and development of a Java-based LEGv8 simulator.

\section*{Some context}
\paragraph{}
LEGv8 is an instruction set architecture (which from now on will be referred to as ISA) designed by D.A. Patterson and J.L. Hennessy to serve as a pedagogical tool in their textbook \emph{Computer Organization and Design
ARM Edition} \cite{patterson2016computer}.
An ISA is a set of specifications that provides a logical description of a processor-based system. It offers a layer of abstraction that both the hardware and the software need to conform to in order to ensure proper interoperability, meaning that code written for a certain ISA is sure to work on any processor designed following the same specification and vice versa.
As the title of the book suggests, the text uses the ARM ISA (more specifically, the v8 version) as a real life showcase of the modern approaches to the design of computer systems. However, in some of the chapters, a simpler architecture is needed to better present some of the material, and for this reason the authors identified a simpler and more coherent subset of the ARMv8 ISA which they named LEGv8. 
Being a fictional ISA created to serve the pedagogical purposes of a single textbook, there are currently no commercially available hardware implementations of the architecture defined by LEGv8, meaning that any code written for it can only be executed by the means of software simulation.

\paragraph{}
Being a relatively popular textbook in undergraduate Computer Architeture courses, an ecosystem of simulators has grown over the years mostly from the enterprise of students and professors, either because of personal interests or because of course requirements. These simulators can be divided in two categories: hardware simulators and software simulators. The former concern themselves with simulating the inner workings of a hypothetical physical implementation of the LEGv8 ISA, while the latter are more interested in simulating the logical execution of the code from the point of view of an external user. Regardless of the type, none of them achieve a complete simulation of the entire LEGv8 ISA for one reason or another, especially when dealing with floating point operations. This might not be a issue when wanting to execute, test or showcase code limited to only a few of the available instructions, as would be the case in a university course with well defined requirements, but problems arise when a student or educator might want to do so using the full spectrum of operations specified by LEGv8. For this reason, a simulator capable of achieving a complete and correct coverage of the entire ISA would be a welcomed addition in the current LEGv8 software landscape.

\paragraph{}
Among the software simulators available online, one is being developed and published by Arm itself \cite{legv8simARMrepo} and is the subject of this thesis' work. \\
Arm's LEGv8 simulator is almost entirely written in Java and uses the GWT framework \cite{gwtweb} to distribute the software in the form of a web application. Its codebase is well structured and its use of a platform-agnostic and high level language allows for easy development and extension. Furthermore, the use of GWT gives it the ability to be run on any device running a modern-ish version of any HTML5-compatible web browser. Due to GWT's integration with the Eclipse IDE for Java, this development environment was chosen for the project. These design choices, however, present some criticalities which will be discussed later. \\
Other issues arise regarding the execution of the simulator, which presents critical bugs that impede any kind of conditional statements, loops, and that don't allow for the execution of user defined functions. It is clear then that a simulator only capable of executing branchless code without support for subroutines is not adequate for the simulation of the vast majority of computer code. \\
As mentioned in the previous paragraph, none of the publicly available simulators cover the entire LEGv8 ISA, and Arm's one is no exception. This, combined with the presence of the aforementioned simulation-breaking bugs, might discourage its usage and even question its utility. Nonetheless, because of its popularity and the nature of its publisher, it has been deemed the highest impact project when deciding which one to work on for this thesis. 

\section*{Presentation of the thesis' work}
\paragraph{}
The work presented in this thesis has tackled different parts both internal and external to the simulator and can be divided into 5 logical phases, not necessarily in chronological order.

\paragraph{0. Study of the simulator.}
The initial part of the work has consisted in analyzing the structure and inner workings of the simulator in order to identify the developers' intentions with its design and the source of its breaking bugs. An important part of this work has been understanding the codebase's undocumented build process and dependencies, documenting them through a comprehensive tutorial, and making it publicly available to allow for future development and collaboration.

\paragraph{}
An overview of the LEGv8 ISA, a survey of the available simulators, and an overview of Arm's codebase together with its set-up shortcomings and the work done fixing them will be the subject of Chapter \ref{chap:chap1}.

\paragraph{1. Modernization of the development environment.}
A few of the design choices made by the original developers have had some unfortunate consequences on the current state of the project. First among them has been the lack of documentation regarding the build process for the simulator, which has made it impossible to effectively develop fixes and enhancements to the codebase and be able to deploy them and share them. Secondly, the choice of GWT as a framework upon which to base the project has caused the developers to depend on an outdated version of the Eclipse IDE for development. Furthermore, there have been some changes to newer versions of GWT that updated how the library is deployed and break the original Eclipse's workflow.\\
The work done in this phase has been to update GWT to its latest iteration and to adapt the codebase to accomodate the new build environment. This was achieved by integrating Maven \cite{mavenweb} into the project, thus bringing it to a modern standard of build automation and dependency management. Thanks to this work, the simulator's development environment has been decoupled from Eclipse and can now be set-up in a few simple steps both from the terminal or via any Maven-compatible Java IDE. This allows for an almost automatic set-up process and opens the doors to future development and collaboration thanks to the ability to utilize different and more featureful IDEs, new language features, and take advantage of the automations provided by Maven such as code testing and plugins.

\paragraph{}
A more extensive overview of the project's structure, its libraries, configuration, and the steps taken to modernize it will be included in Chapter \ref{chap:chap2}.

\paragraph{2. Critical bug fixes.}
As mentioned before, the simulator presented some breaking bugs that impeded the expected functioning of the code execution.\\
The first bug consisted in an error in the logic of the simulator that incorrectly evaluated comparisons between integers. Since integer comparisons are at the foundation of conditional instructions and loops, this caused branches to not be taken and loops to be exited prematurely.\\
The second bug was instead the result of a problem in setting the correct return position when calling a user defined function. This meant that when a function was executed, once the final instruction was reached, instead of returning to the main program where the function was called, the program entered the function again in an infinite loop.

\paragraph{}
The nature of these bugs and the steps taken to fix them will be discussed in Chapter \ref{chap:chap3}.

\paragraph{3. Completing the integer arithmetic and integrating floating point support.}
From the survey of available simulators, Arm's didn't distinguish itself for its coverage of the LEGv8 instruction set. This is both in the case of the integer instructions (having implemented only a subset) and the floating point ones (none of which have been implemented).\\
Regarding the addition of the missing integer instructions, the structure of the codebase allowed for an easy implementation, although some limitation of the Java language necessitated alternative solutions. None of the instructions fundamentally changed the inner workings of the simulator, thus all that was neeeded was to imitate the already implemented logic.\\
The floating point logic, on the other hand, introduced new elements to the design of the simulator since it required separate places in which to store the values being operated with, and didn't allow the intermixing of integer and floating point values and instructions. In order to implement these features, some previously untouched parts of the simulator have been extended with the necessary logic.

\paragraph{4. Updating the simulator's UI.}
All the changes and additions mentioned in the previous paragraph needed to be presented to the user through an update of the original UI.\\
An initial addition, independent from all the changes already mentioned, was the introduction of the visualization of the main memory of the simulated computer. This allows the user to see the complete state of of the memory in real time, both for didactic purposes and to check for errors in the code.\\
Secondly, all the new floating point logic required the addition of new elements in the UI to visualize the current floating point values in the working memory divided between single and double precision and with the correct dotted decimal notation.\\
Lastly, a general reorganization of the UI elements was performed to make the simulator fit better on an average computer monitor or projector and present more information without navigating the page.

\paragraph{}
A complete overview and explanation of all the introduced changes and additions, both to the internal logic of the simulator and its UI, will be provided in Chapter \ref{chap:chap4}.

\paragraph{}
The work presented in this thesis only tackled a subset of the problems and shortcomings of the simulator, as they are numerous and would require a great amount of additional work as many of them are intrinsic to the design choices of the original developers. The concluding chapter will include a final overview of the work done and a discussion the possible further developments of the simulator and how they might be achieved.
\newpage

%*******************************************************
% Chapters
%*******************************************************
\mainmatter

% Do not modify this part
\renewcommand{\chaptermark}[1]{\markboth{\MakeUppercase{\chaptername\ \thechapter.\ #1}}{}}
% Do not modify this part

% Add all the necessary chapters. One can add all the chapters
% inside of a single .tex file or through one .tex file per chapter.

\chapter{}

\epigraph{"If 10 years from now, when you are doing something quick and dirty, you suddenly visualize that I am looking over your shoulders and say to yourself: 'Dijkstra would not have liked this', well that would be enough immortality for me."}{\textit{Edsger W. Dijkstra}}

\section*{The LEGv8 simulators landscape}

The current landscape of publicly available LEGv8 simulators can be divided into two categories: simulators that aim to reproduce the logical design presented in the textbook in chapter ?, and the simulators providing a high level simulation of the instruction set as defined in the book.
The survey was performed on GitHub using ``LEGv8'' and ``simulator'' as keywords and only those in a reasonably working state (as per the author) have been considered.

\subsection*{Software simulators}

\begin{table}[H]
	\centering
	\resizebox{\columnwidth}{40}{\small \begin{tabular}{|c|c|c|c|c|c|c|}
		\hline
		\textbf{Repository} & \textbf{Language} & \textbf{Integer Support} & \textbf{Pipelined} & \textbf{Registers view} & \textbf{Stack view} & \textbf{Floating Point Support} \\
		\hline
		\url{https://github.com/lcpckp/leg-cpu-sim} & Java & Partial & No & Yes & Yes & No \\
		\hline
		\url{https://github.com/chrwoods/legv8-emul} & C/C++ & Partial & Yes & Yes & Yes & No \\
		\hline
		\url{https://github.com/mtalyat/LEGv8Day} & C\# & Partial & No & Yes & Yes & No \\
		\hline
		\url{https://github.com/eaxworthy/LegV8Interpreter} & Python & Partial & No & Yes & Yes & No\\
		\hline
		\url{https://github.com/AdinAck/LEGv8-Simulator} & Swift & Partial & No & Yes & Yes & No \\
		\hline
		\url{https://github.com/anvitha305/legv8sim} & Python & Partial & No & Yes & Yes & Double precision only \\
		\hline
		\url{https://github.com/dangbandy/LegV8-Simulator} & C++ & Partial & No & Yes & Yes & No\\
		\hline
		\url{https://github.com/schang412/LEGv8-PyEmu} & Python & Partial & No & No & No & No\\
		\hline
		\url{https://github.com/GeorgePerreault/LEGV8_Interpreter} & Python & Partial & No & Yes & Yes & No\\
		\hline
	\end{tabular}}
	\caption{The surveyed software simulators}
\end{table}

They utilize high level languages such as C++, Python, Swift, TypeScript and Java. Some of them offer a graphical interface, pipelined execution and none of the implement the LEGv8 ISA in its entirety.



\subsection*{Hardware simulators}

\begin{table}[H]
	\centering
	\resizebox{\columnwidth}{45}{\small \begin{tabular}{|c|c|c|c|c|c|c|}
		\hline
		\textbf{Repository} & \textbf{Language} & \textbf{Integer Support} & \textbf{Pipelined} & \textbf{Floating Point Support} \\
		\hline
		\url{https://github.com/nxbyte/ARM-LEGv8} & Verilog & Partial & Yes & No \\
		\hline
		\url{https://github.com/phillbush/legv8} & Verilog & Partial & Yes & No \\
		\hline
		\url{https://github.com/ronitrex/ARMLEG} & Verilog & Partial & Yes & No \\
		\hline
		\url{https://github.com/mattco98/LEGv8-Processor} & Verilog & Partial & Yes & Partial \\
		\hline
		\url{https://github.com/amaurilopez90/LEGv8-CPU} & Verilog & Partial & Yes & No \\
		\hline
		\url{https://github.com/miguelangelo78/LEGv8-ISA} & Verilog & Partial & Yes & No \\
		\hline
		\url{https://github.com/brianworts/LEGv8_SingleCycle_Processor} & Verilog & Partial & Yes & No \\
		\hline
		\url{https://github.com/egflo/LEGv8} & Verilog & Partial & Yes & No \\
		\hline
		\url{https://github.com/ad153153/LegV8} & Verilog & Partial & Yes & No \\
		\hline
	\end{tabular}}
	\caption{The surveyed hardware simulators}
\end{table}

They use mostly Verilog as their hardware description language and implement an incomplete subset of the LEGv8 ISA. Some of them follow closely the design of the textbook while others expand upon it adding more executable instructions. None of them offer a graphical interface nor implement the ISA in its entirety.
\\

It is clear from this brief survey that the LEGv8 simulators space lacks any desirable candidates for code execution and inspection, as the software simulators are incomplete and platform-dependant, and the hardware ones lack interactivity and comprehensive visual output capabilities.

\section*{ARM's LEGv8 simulator}

This is the simulator officially provided by ARM Education and is the subject of this thesis' work. It is written in Java 8 and uses Google's GWT framework to transpile the code into native JavaScript to allow the simulator to be executed inside a web browser as a normal web application. It provides a comprehensive user interface displaying an interactive text editor (provided by AceGWT) to input LEGv8 code and to display errors, and a visualization of the state of the \emph{X} registers.
When selecting the single-cycle execution mode, a visualizaton of the logical scheme of the LEGv8 ISA is presented and for each step of the execution various components change color to indicate the current stage of the pipeline. For the pipelined execution mode, the visualization is slightly modified to include pipeline-specific information such as pipeline registers, the hazard detection unit and the forwarding unit. An additional textual representation of the pipeline is provided to see the stage occupied by each instruction at any given moment.

\begin{figure}[H]
	\centering
	\subfigure[Single cycle]{
		\includegraphics[width=.85\textwidth]{img/old_single_cycle.jpeg}
	}
	
	\subfigure[Pipeline]{
		\includegraphics[width=.85\textwidth]{img/old_pipeline.jpeg}
	}
	\caption{The simulator's main page with the two different execution modes}
\end{figure}

\subsection*{Features}

This simulator presents many favorable characteristics:

\begin{itemize}[label=\textendash]
	\item Written in Java (platform agnostic, extensible)
	\item Compiled as a web application (platform agnostic and easily deployable)
	\item Embedded text editor to input code with, and error display
	\item Clear and rich visualization of the \emph{X} and flag registers and the datapath of the CPU
	\item Almost all of the integer arithmetic is already implemented
	\item All types of integer \emph{LOAD} and \emph{STORE} instructions are already implemented, including \emph{STXUR} and \emph{LDXUR}
	\item Officially distributed by ARM Education (biggest support and discoverability)
\end{itemize}


\subsection*{Problems}

Unfortunately many problems present themselves when trying to run or develop the simulator:

\begin{itemize}[label=\textendash]
	\item Absence of any documentation on how to build the project and design choices behind it
	\item Executable version distributed in automatically-generated web page form
	\item Pipeline execution is incomplete
	\item The mechanism for calling subroutines is broken and results in infinite loops, making it impossible to delegate code to other functions
	\item The mechanism for performing comparisons is broken and results in the wrong branches being taken, making it impossible to perform conditional operations and loops
	\item The project is heavily dependent on the Eclipse Java IDE with an old GWT plugin to perform the build process
	\item The project depends on the outdated and barely supported GWT library to deploy the simulator as a web application. This restricts the developers from using newer Java features or better web frameworks.
\end{itemize}


I present below a demonstration of the bugs regarding the subroutine calls and number comparisons:

\begin{figure}[H]
	\centering
	\subfigure[BL instruction writes the incorrect address to the return register (LR)]{\includegraphics[width=.45\textwidth]{img/br_bug_1.png}}
	\subfigure[Jumps to the subroutine and increments X0]{\includegraphics[width=.45\textwidth]{img/br_bug_2.png}}
	\subfigure[Reads wrong address from LR]{\includegraphics[width=.45\textwidth]{img/br_bug_3.png}}
	\subfigure[Returns to the start of the subroutine instead of the main program]{\includegraphics[width=.45\textwidth]{img/br_bug_4.png}}
	\caption{Branch returns to the wrong instruction, making it execute the branch in a loop}
\end{figure}

\begin{figure}[H]
	\centering
	\subfigure[X0 < X1]{\includegraphics[width=.45\textwidth]{img/cmp_bug_1.png}}
	\subfigure[Comparison sets the flags incorrectly]{\includegraphics[width=.45\textwidth]{img/cmp_bug_2.png}}
	\subfigure[Branch gets ignored]{\includegraphics[width=.45\textwidth]{img/cmp_bug_3.png}}
	\subfigure[Wrong instruction executed]{\includegraphics[width=.45\textwidth]{img/cmp_bug_4.png}}
	\caption{Comparisons do not set the correct flags and thus fail}
\end{figure}

\subsection*{Motivations}

For these reasons, this simulator was chosen as the subject of my thesis:

\begin{itemize}[label=\textendash]
	\item Maximize the impact of my work by fixing and improving the most popular simulator available
	\item Provide the first complete implementation of the LEGv8 instruction set
	\item Allow the Digital Systems Architecture course at UniTS and other courses in general to have a working LEGv8 simulator for more effective teaching
	\item Opportunity to work on a real Java code base
\end{itemize}
\newpage
\chapter{Sit Amet}\label{chap:sit}

\epigraph{"Much of the excitement we get out of our work is that we don't really know what we are doing"}{\textit{Edsger W. Dijkstra}}

\lipsum[1]

\section{Lorem Ipsum}
\lipsum[2-4]

\subsection{Dolor sit amet}
\lipsum[5-7]
\newpage
\chapter{Lorem Ipsum}\label{chap:lipsum}

\epigraph{"Program testing can be used to show the presence of bugs, but never to show their absence!"}{\textit{Edsger W. Dijkstra}}

\lipsum[1]

\section{Dolor}
\lipsum[2-4]

\subsection{Sit amet}
\lipsum[5-12]
\newpage
\chapter{Current problems and further development}\label{chap:chap4}

\epigraph{``Perfecting oneself is as much unlearning as it is learning.''}{\textit{Edsger W. Dijkstra}}



\subsection*{Current shortcomings and proposals}

Here is a list of things that can be fixed or added without fundamentally changing how the project is structured:

\begin{itemize}
	\item The pipelined execution does not work properly, especially with the newly introduced floating point operations. This is because the code modeling the CPU is not logically divided into the pipeline stages and makes it difficult to run and synchronize multiple instructions at a time. This can be done by reorganizing the project's code to make it follow closer to the 5 stages of the LEGv8 pipeline.
	\item Even though the project is written in Java, it makes many design decisions that don't make use of the power of object-orientation. It doesn't properly divide the components of the ISA into their own classes, it presents many code repetitions and doesn't use many abstractions, and in general some classes and methods are disproportionally large. The code base should be refactored using the latest design principles of Java's object-oriented programming and make use of the additions that GWT has implemented from v2.7 onwards.
	\item The project lacks any code testing capabilities. Tests should be written using Maven's convention in order to make sure the simulator works correctly and to avoid breaking changes in the future.
	\item The project currently has to include a custom-compiled version of AceGWT as a local repository. This can be fixed by either:
	\begin{itemize}[label=$\rightarrow$]
		\item Uploading this custom version to Maven's central repository.
		\item Configuring Maven and using some plugins in order to use AceGWT's GitHub repository as a Maven repository and automatically apply the patches and build the custom version of the library on the fly.
	\end{itemize}
	\item Make the web UI more responsive and change its layout to work better on devices with smaller screens. In general, improve the look and feel of the application.
	\item Currently, code can only be executed by manually stepping over the instructions. A way of automatically running all the code until completion or stepping automatically with a user-defined clock speed could be helpful when trying to test results faster.
	\item The project lacks documentation. More code comments should be written and a more thorough description of the simulator's design and functionalities should be provided both to users and to developers. This thesis could be a start.
\end{itemize}

\subsection*{Structural problems}

These are the problems I found with the simulator that, should they be fixed, would require a lot of work and would change the code base in a fundamental way.

\begin{itemize}
 	\item The textbook doesn't really specify the logical implementation of the ISA for anything other than a few integer instructions. This means that, for example, things like ALUop codes are not defined for floating point operations. In order to provide a complete simulation of the LEGv8 ISA, some arbitrary design decisions should be made to extend Patterson's and Hannessy's work.
 	\item Similarly, the texbook doesn't talk about pipelined execution in the context of floating point operations. This means that implementing it would require the programmer to make its own informed design decisions.
 	\item Java does not allow data structured to contain more than $2^{32}$ elements. This means that things such as the main memory cannot be properly simulated with a single data structure, but a tiered approach should be taken. Of course this is a very minor problem, since LEGv8 is purely for educational usage and realistically no program using more than a moderate amount of memory and instructions will be written.
 	\item If the application needs to be deployed and ran in the browser as a web application, replacing GWT could be considered. It's a very old library that has stopped being officially supported by Google long ago \footnote{\url{https://en.wikipedia.org/wiki/Google_Web_Toolkit}} and cannot keep up with the new features of both the Java language and the web. A new framework for creating self-contained web applications with Java should be identified and the project rewritten to make use of it. Unfortunately, GWT's main advantage is the ability to write Java code and have it seamlessly transpiled to JavaScript, whereas most frameworks use Java only for the back end and require the programmer to write the client-side logic in JavaScript. If no such alternative framework exists, modernizing the code base as much as possible within GWT's constraints or surveying other programming languages  could be taken into consideration. Alternatively, thanks to the portability of Java, a native UI could be written and the simulator distributed through a \verb|.jar| executable.
\end{itemize}

\newpage

%*******************************************************
% Conclusions
%*******************************************************
\chapter{Concluding remarks}
% Do not modify this part
%\addcontentsline{toc}{chapter}{Conclusions}
\label{chap:concl}
\markboth{CONCLUSIONS}{CONCLUSIONS}
% Do not modify this part

\epigraph{``Probably I am very naive, but I also think I prefer to remain so, at least for the time being and perhaps for the rest of my life.''}{\textit{Edsger W. Dijkstra} -- EWD923A}

\paragraph{}
This thesis' work has consisted in studying Arm's official LEGv8 simulator, reconstructing and documenting its build process, bringing it up to an acceptable working state by fixing critical bugs in the codebase, and extending it by implementing the entirety of the LEGv8 ISA in single cycle mode and improving its functionalities and UI. As of this date, this is the only publicly available simulator to offer a complete implementation of LEGv8 in single cycle mode and a comprehensive visualization of the stack, of all the registers, and the datapath, as shown in the Patterson's and Hannessy's book \cite{patterson2016computer}.
\paragraph{}
Not every part of the simulator has been fixed or touched upon. Because of the undergraduate nature of this thesis, there wasn't enough time to tackle all the problems still present in the codebase. In this concluding chapter, we will provide a comprehensive list of issues and missing features that can be still worked on without fundamentally changing the nature of the project. In the end, we will discuss the more structural problems that would require a similar amount of effort as this thesis' to fix, and which could be considered suggestions for future developments.

\section{Current shortcomings and missing features}
\paragraph{}
In no particular order, we present a few observations about some features, both of the original simulator and the ones stemmed from this thesis' work, that could definitely be improved in incremental iterations.
\subsection{Making full use of Maven}
\paragraph{}
The introduction of Maven into the project has already given it a significant boost in configurability, system agnosticism, and ability to be automated in all of its aspects. Maven is a complicated and state of the art build system for Java, and its configuration is not straight forward. Many steps of of its integration in the project might have been performed incorrectly in this thesis, or without following the best practices. For these reasons, improving upon the configuration of Maven, by making use of plug-ins or by cleaning up the code, would make it even easier to deal with the set-up and build processes.
\subsection{Updating AceGWT and giving it a new home}
\paragraph{}
The updated version of AceGWT used in the new version of the project doesn't currently reside in the Maven central repositories nor in the official GitHub repository. An effort of informing and helping the original authors to officially integrate this new update, would allow for the removal of a manual, local installation of the library in the user's filesystem. Of course, updating the Ace editor itself to a newer version and creating the new bindings to GWT would be a welcome change, although it would probably require as much work as the original authors'.
\subsection{Refactoring the codebase}
\paragraph{}
Even though it has been repeated many times in this dissertation that the Java codebase is quite well structured, its code in many places still resembles more of a C-like design. Making too much use of primitive types and not really making use of all the OOP capabilities of Java makes things more complicated than necessary. Since GWT 2.11 started supporting Java 11 features, reorganizing the code to try and make use of these new tools might make future development less cumbersome and the code more extensible.
\subsection{Creating more documentation}
\paragraph{}
In this dissertation we have tried to include as much information as possible about the simulator, but the ultimate arbiter of truth remains the implemented code. For this reason, understanding the inner workings of the simulator and providing additional documentation in the form of design documents, comments, and Javadoc, might make life easier for future developers. This thesis could be taken as a starting point for this effort.
\subsection{Further improvements to the UI}
\paragraph{}
The current state of the UI and UX is quite functional, but, although GWT is an old and outdated framework for creating web applications, it is still possible to make some improvements, such as adding a true responsive design to the web page, especially to make it more usable with mobile devices. Also, a rethinking of the UI elements might allow for a better visual design, such as giving different registers different colors or providing the state of the pipeline with a visual representation, instead of the textual one in the CPU log. Lastly, the stack could be made dynamic by allowing the user to manually browse through the entire memory, either by searching for a specific address or moving around using navigational arrows or scroll bars.
\subsection{Improving the LEGv8 development experience}
\paragraph{}
Ace is a powerful web editor, and as such it can be thoroughly configured. Things such as LEGv8 syntax highlighting could be added by specifying the LEGv8 grammar in an appropriate format and feeding it to the editor. Another shortcoming is that, between page reloads, the contents of the editor get flushed. Introducing permanence to the written code or the ability to upload LEGv8 assembly via a text file, would make resuming development or showcasing examples much easier. Also, being able to choose to run the code until completion -- either instantly or using a custom clock speed -- instead of manually going over each instruction, would help in this aspect.
\subsection*{Testing the logic}
\paragraph{}
Maven makes it easy to implement a testing suite for the simulator. Creating an automatic battery of manual or parameterized tests to be ran each time new changes are made, would make it not only simpler to test the current coverage and correctness of the implemented instructions, but would allow for faster development going forward.

\section{Structural problems and future developments}
\paragraph{}
The points that will be discussed now are more fundamental and deeply embedded into the original design choices of the developers, and would require much more work to be implemented than the ones listed in the last section. 
\subsection{Extending the LEGv8 ISA}
\paragraph{}
As was discussed in Chapter \ref{chap:chap1}, LEGv8 is defined in two ways throughout Patterson's and Hannessy's book \cite{patterson2016computer}. The more detailed one is only reserved to the minimal working example presented in their Chapter 4, while the more abstract one is the one used in the simulator to implement the instructions. Since Arm's simulator provides such a comprehensive representation of the instruction and the datapath, extending the detailed specification of LEGv8 to cover the entire ISA would be of great pedagogical help. This operation would require the developer to make informed decisions in order to be consistent with the textbook's exposition. Something like this was already done, in a much smaller scale, in the work presented in this thesis, when talking about the jump conditions for floating-point operations, where we decided to use ARMv8's convention.
\subsection{Expanding and fixing the pipeline}
\paragraph{}
The pipeline execution mode was not touched upon in this dissertation. Even when limiting ourselves to integer-based instructions, there are many cases in which it stalls or the execution behaves in unexpected ways. Furthermore, it's completely unequipped to deal with the new floating-point logic. Unfortunately, the \verb|CPU.java| class is not designed as a pipeline, as it's not logically divided into the proper datapath components that make up the processor. Restructuring the simulator, especially with Java's OOP capabilities, to more closely resemble the five stages, would create a much simpler design that would automatically work in pipeline and single cycle mode. This operation would require major changes to many classes inside the project, and it should take inspiration from the implementations done in the hardware simulators showcased in Chapter \ref{chap:chap1}.
\subsection{Farewell, GWT?}
\paragraph{}
As Appendix \ref{chap:appA} shows, dealing with GWT can be cumbersome. Even though, from the build process point of view, this has been solved in Chapter \ref{chap:chap2}, the entrenched obsolescence of GWT remains. The web is a place of constant innovation, and much better frameworks for creating web applications have been developed since the days of GWT. For this reason, especially when thinking about longer timescales, new solutions should be considered, starting with new libraries for Java. If none are considered suitable, perhaps other languages could be considered, in which case a complete rewrite and redesign would need to happen. It is clear that this last suggestion creates a sort of ship of Theseus situation, and should be the last one to be considered for future developments.



\newpage

%*******************************************************
% Appendix
%*******************************************************
% Do not modify this part
\pagestyle{fancy}
\renewcommand{\chaptermark}[1]{\markboth{\MakeUppercase{APPENDIX\ \thechapter.\ #1}}{}}
% Do not modify this part

\appendix
\label{appendix}

\chapter{Walkthrough of the original project set-up}\label{chap:appA}
\paragraph{}
In this appendix, we will lay out a more detailed and visual retelling of the steps needed to build the original codebase using the original build methods. After the original build process was reconstructed, a 50-page document, complete with screenshots, was compiled and published inside a pull request to Arm's repository \cite{web:setuptutorialoriginal}, both in PDF and HTML format. This appendix is an attempt to provide a more compact version of this tutorial, in order to embed it into the thesis without requiring access to external documents.
\section{Downloading the resources}
\paragraph{}
This section will contain all the links to the resources needed before the set-up process can begin. We will focus on downloading as much as possible locally, as these web resources might one day be taken offline. In case that happens, a directory containing all the resources is all that will be needed to reproduce the steps. This tutorial will suppose all the downloads are extracted and put inside a \verb|/sources| folder somewhere in the user's filesystem.
\subsection{JDK 8}
\paragraph{}
As was mentioned in Chapter \ref{chap:chap2}, JDK 8 is our best option, even though, as the changelogs for version GWT 2.7 \cite{gwt2.7web} show, Java 8 syntax was supported only starting from version 2.8. The choice of JDK vendor is not really relevant, as the features emulated by GWT are the elementary ones of the language. In this case, it was chosen to use Amazon's Corretto 8 JDK \cite{web:amazoncorretto8}. It can be either installed on the operating system or downloaded as a compressed folder, with the latter being recommended for this procedure.
\subsection{Eclipse IDE}
\paragraph{}
Due to some library changes to newer versions of Eclipse, the last one compatible with the GWT plug-in used in this thesis's work is currently version 2023-09 \cite{web:eclipse202309}. Download and extract the \verb|Eclipse IDE for Java Developers| version.
\subsection{GWT plug-in for Eclipse}
\paragraph{}
The GWT 4.0.0 plug-in repository can be downloaded and extracted locally from the GWT's plug-ins GitHub Releases page \cite{web:gwtplugin4.0}.
\subsection{GWT 2.7 and DTD 2.7}
\paragraph{}
From the releases page on GWT's website \cite{web:gwtreleases}, we can download and extract the 2.7 version. We must also download the corresponding DTD file \cite{web:gwtdtd2.7}.
\subsection{AceGWT and DTD 2.8.2}
\paragraph{}
Similarly, we can download GWT 2.8.2's DTD \cite{web:gwtdtd2.8.2}. To download AceGWT we can simply go to the Releases page on their GitHub repository \cite{web:acegwtgit} and download the 1.0.0 release.
\subsection{Arm's LEGv8 simulator}
\paragraph{}
The last component we need is the simulator itself. Like AceGWT, it can be downloaded from the Releases section of its GitHub repository \cite{legv8simARMrepogit} and extracted.
\section{Setting everything up}
\paragraph{}
Now everything is in place inside the \verb|/sources| folder to begin the set-up process.
\subsection{Running Eclipse}
\paragraph{}
The Eclipse version that has been downloaded is in executable form, without an installation. We can run it and, when asked, we can choose our preferred folder for the Eclipse workspace.
\subsection{Pointing to the JDK}
\paragraph{}
Now we can tell Eclipse to use the downloaded JDK by navigating to the top taskbar and selecting \verb|Window| $\longrightarrow$ \verb|Preferences| $\longrightarrow$ \verb|Java| $\longrightarrow$\newline \verb|Installed JREs|. Now we can press the \verb|Add...| button and tell Eclipse it's a \verb|Standard VM|. After pressing \verb|Next|, under the \verb|JRE home| label, we select \verb|Directory...| and put the root folder where the JDK we have downloaded in present. Since we have downloaded the JDK as a compressed file, some additional folders might have been created. Start from the outer one and go ony by one until the \verb|JRE system libraries| list gets filled with elements. Now we can \verb|Finish|, select the newly added JDK from the list, and \verb|Apply and Close|.
\subsection{Installing the GWT plug-in}
\paragraph{}
On the top taskbar, go to \verb|Help| $\longrightarrow$ \verb|Install New Software...| and press \verb|Add...|. Under the \verb|Name| label, press \verb|Local...| and point to the extracted repository folder and press \verb|Add...|. Select the \verb|GWT Eclipse Plugin| entry and press \verb|Next >|. Continue with the installation, accept the license, and finish. A pop-up window will appear asking to trust the following content. Press \verb|Select All| and \verb|Trust Selected|. The same window will appear again, but this time asking to trust the artifacts. Do the same procedure, and restart Eclipse.
\subsection{Importing the simulator}
\paragraph{}
We can now add the simulator's sources to the workspace by going to the top taskbar and navigating to \verb|File| $\longrightarrow$\newline \verb|Open Projects from File System...|. Under the \verb|Import source| label, press on \verb|Directory...| and select the folder where the simulator has been extracted to. Two elements will appear on the list, select only the one with \verb|Eclipse project| under the \verb|Import as| column and press \verb|Finish|. If Eclipse asks about marketplace solutions or project natures, decline.
\subsection{Configuring GWT 2.7}
\paragraph{}
The IDE will complain about a lack of resources. First among them, the GWT SDK. Right click on the \verb|LEGv8_Simulator| folder inside the package explorer on the left, and navigate to \verb|Build Path| $\longrightarrow$\newline \verb|Configure Build Path...|. Now go to the \verb|Libraries| tab and press\newline \verb|Add Library...|. Choose GWT as the type of library to be added, and in the next window select \verb|Configure SDKs...|. Press \verb|Add...|, then under the \verb|Installation directory| label press \verb|Browse...| and select the folder where GWT 2.7 was extracted, and press \verb|OK|, \verb|Apply and Close|, and \verb|Finish|. If asked about synchronization, refuse. At this point we will find ourselves back to the Libraries tab, and we can remove the old missing GWT SDK by selecting it and pressing \verb|Remove|.
\subsection{Adding AceGWT}
\paragraph{}
The last step we need to perform is to include the AceGWT library inside the project in the correct location. While still in the \verb|Build Path| window, go to the Projects tab, select the AceGWT entry and remove it. Go to the Sources tab, select the missing \verb|LEGv8_Simulator/test| entry, and remove it as well. Now press \verb|Apply and Close|. Right click on the \verb|/src| folder inside the package explorer and go to \verb|Import...| $\longrightarrow$ \verb|General| $\longrightarrow$ \verb|File System| and press \verb|Next|. Now, under the \verb|From directory| label, press \verb|Browse...| and select the \verb|/src| folder inside of the AceGWT folder inside of where AceGWT's release was extracted. Select the newly added  \verb|src| entry in the left panel of the window, and press \verb|Finish|.
\subsection{Pointing to the DTDs}
\paragraph{}
It is now time to make use of the DTD files we have downloaded for both GWT 2.7 and 2.8.2. First, open with a text editor the\newline \verb|LEGv8_Simulator.gwt.xml| file inside the \verb|/src/com.arm.legv8simulator| package in the package explorer, and substitute the\newline \verb|http://gwtproject.org/doctype/2.7.0/gwt-module.dtd| string inside the \verb|DOCTYPE| definition with the absolute path of the 2.7 DTD in your filesystem. Now open the \verb|AceGWT.gwt.xml| file in the \verb|edu.ycp.cs.dh.acegwt| package. Since this xml file lacks the \verb|DOCTYPE| definition, simply copy and paste the one from the other file in the same location, and substitute the path with the one for the 2.8.2 DTD.
\subsection{Compiling the project}
\paragraph{}
To compile the project we can navigate to the second taskbar from the top, and press the newly added GWT button (the red one bearing the GWT logo) and select \verb|GWT Compile Project...|. Under the \verb|Project| label we press \verb|Browse...|, select the simulator's project, and press \verb|OK|. We can now choose our preferred options for the \verb|Log level| and \verb|Output style|. The entry point for GWT will already be selected, as there is only one. If we press \verb|Compile|, the build process will begin, with the resulting output being put inside the \verb|/war| folder that will have appeared in the package explorer.
\section{New developments}
\paragraph{}
During the writing of the thesis, in the second half of 2024, the GWT community has developed an updated version of the GWT plug-in that fixes the aforementioned compatibility problems with newer versions of Eclipse. This 4.1.0 version is not yet stable and can be found in the same GitHub repository \cite{web:gwtplugingit} in the Releases section. This update of the plug-in includes the latest 2.11 version of GWT but removes the inclusion of version 2.7, making some changes to the naming conventions and other smaller details. For these reasons, even though the set-up process of the original codebase could probably now be performed in an even more expedite manner, some changes would need to be made to the tutorial, and things might break in new, subtle ways. Thus, it was decided to keep this version of the tutorial, since it serves an archival purpose and the new build process using Maven should be used anyway.
\chapter{Guide for the new Maven-based set-up}\label{chap:appB}
\paragraph{}
In this brief appendix, we will provide the tools to build the updated version of AceGWT and some tutorials on how to use the updated build system on a variety of Java IDEs supporting Maven.
\section{Maintaining AceGWT}
\paragraph{}
As was touched upon in Chapter \ref{chap:chap2}, in order to update AceGWT to the latest GWT version, some changes had to be made. Unfortunately, they are currently not present in the official GitHub repository \cite{web:acegwtgit}, and thus a custom version of the library needs to be built and used. Some additional documentation about the project is available on the original repository's wiki.
\paragraph{Creating parent pom.xml}
First of all, to aid the build process, it is useful to create a pom.xml file in the root of the project. This is reported in Listing \ref{lst:acegwtparentpom}.
\begin{lstlisting}[float, language=XML, label={lst:acegwtparentpom}, caption={The new parent pom.xml}]
<?xml version="1.0" encoding="UTF-8"?>
<project xmlns="http://maven.apache.org/POM/4.0.0"
         xmlns:xsi="http://www.w3.org/2001/XMLSchema-instance"
         xsi:schemaLocation="http://maven.apache.org/POM/4.0.0 http://maven.apache.org/xsd/maven-4.0.0.xsd">
  <modelVersion>4.0.0</modelVersion>
  <groupId>edu.ycp.cs.dh</groupId>
  <artifactId>acegwt-2.11.0</artifactId>
  <version>1.3.3</version>
  <packaging>pom</packaging>
  <name>AceGWT-2.11.0</name>
  <description>
    Ace is an embeddable code editor written in JavaScript. It matches the features and performance of native editors such as Sublime, Vim and TextMate. It can be easily embedded in any web page and JavaScript application. Ace is maintained as the primary editor for Cloud9 IDE and is the successor of the Mozilla Skywriter (Bespin) project. AceGWT is an integration of Ace into GWT.
  </description>
  <modules>
    <module>AceGWT</module>
    <!-- <module>AceGWTDemo</module> -->
  </modules>
</project>
\end{lstlisting}
\paragraph{Updating children pom.xml files}
Due to the new ownership of the GWT project, new artifact names are needed to include GWT 2.11. Listing \ref{lst:acegwtchildrenpom} shows the changes required for both the AceGWT and AceGWTDemo modules.
\begin{lstlisting}[float, language=XML, label={lst:acegwtchildrenpom}, caption={The changes to the children's pom.xml}]
...
<parent>
  <groupId>edu.ycp.cs.dh</groupId>
  <artifactId>acegwt-2.11.0</artifactId>
  <version>1.3.3</version>
</parent>
...
<gwt.version>2.11.0</gwt.version>
...
<dependencyManagement>
  <dependencies>
    <dependency>
      <groupId>org.gwtproject</groupId>
        <artifactId>gwt</artifactId>
        <version>${gwt.version}</version>
        <type>pom</type>
...
<dependencies>
  <dependency>
    <groupId>org.gwtproject</groupId>
    <artifactId>gwt-user</artifactId>
  </dependency>
  <dependency>
    <groupId>org.gwtproject</groupId>
    <artifactId>gwt-dev</artifactId>
  </dependency>
  <dependency>
    <artifactId>gwt-codeserver</artifactId>
    <scope>provided</scope>
  </dependency>
  <dependency>
    <groupId>org.gwtproject</groupId>
    <artifactId>gwt-servlet</artifactId>
    <scope>runtime</scope>
  </dependency>
...
\end{lstlisting}
\paragraph{Updating module.gwt.xml}
The \verb|module.gwt.xml| file, in both children modules, needs only an update to the link of the DTD, as shown in Listing \ref{lst:acegwtdtd}.
\begin{lstlisting}[float, language=XML, label={lst:acegwtdtd}, caption={Updated DTD version}]
<!DOCTYPE module PUBLIC "-//Google Inc.//DTD Google Web Toolkit 2.11.0//EN"
        "http://gwtproject.org/doctype/2.11.0/gwt-module.dtd">
\end{lstlisting}
\paragraph{Building the library}
Now the library has to be built, in our case,  into a .jar file. This can be achieved by running\\ \verb|mvn clean gwt:generate-module compile gwt:package-lib| into the \verb|AceGWT| folder inside the project. This will create a folder called \verb|target|. The contents of this folder are the ones that need to be copied in the local Maven repo folder discussed in Chapter \ref{chap:chap2}.
\section{Set-up tutorials for various Java IDEs}
\paragraph{}
We will present written descriptions of the steps needed to set-up the project in the most popular Java IDEs that support Maven. It is possible to develop the simulator even in environments without Maven support, such as a normal text editor, by editing the code and running Maven via the terminal as showcased in the end of Chapter \ref{chap:chap2}.
\subsection{IntelliJ IDEA}
\paragraph{}
Choose the \verb|Get from VCS| option and copy paste this repository's git link. The Maven project will be cloned into your workspace and the dependencies will be automatically downloaded and configured. After making your changes you can build and package the simulator by going to the Maven panel on the right, navigate to \verb|Graphical-Micro-Architecture-Simulator| $\longrightarrow$ \verb|LEGv8_Simulator| $\longrightarrow$ \verb|Lifecycle| $\longrightarrow$ \verb|package|. The folder containing the simulator will appear under \verb|LEGv8_Simulator/target/|.
IntelliJ Ultimate offers a GWT plugin that warns the programmer when using methods, syntax and classes not implemented by GWT and can generate compile reports. Be warned that in some cases it might show bogous errors (such as missing css directives), but this shouldn't affect the compilation which is done with GWT. If you use IntelliJ without this plugin the errors will disappear but you will not have access to said features.
\subsection{Eclipse}
\paragraph{}
Choose \verb|Import projects...|, select\newline \verb|Git/Projects from Git (with smart import)| $\longrightarrow$ \verb|Clone URI| and copy paste this repository's git link. Keep pressing \verb|Next| until it has finished the procedure. It is recommended to go to \verb|Window| $\longrightarrow$ \verb|Preferences| $\longrightarrow$ \verb|XML (Wild Web Developer)| and enable the download of external resources. Unlike IntelliJ, Eclipse doesn't show directly the package action but has to be added manually by right clicking on the \verb|LEGv8_Simulator| folder and going to \verb|Run As| $\longrightarrow$ \verb|Run Configurations...| and then double click on \verb|Maven Build|. This will create a new configuration for you to edit: give it the name package, select \verb|Workspace...| $\longrightarrow$ \verb|LEGv8_Simulator| as the \verb|Base Directory| and write \verb|package| inside the \verb|Goals| text box. Now you can \verb|Apply| and run it. To run it again just go to the \verb|Run As| menu as before and it should have been added there. The folder containing the simulator will appear under \verb|LEGv8_Simulator/target/| (press F5 in Eclipse to refresh the folders).
Eclipse offers a GWT plugin (which has installation problems with Eclipse versions newer than the 2023-09) that makes compilation easier but, unlike the Maven package action, deploys the compiled sources into a \verb|/war| folder, and doesn't automatically copy-paste the web resources needed to launch the web page. That has to be done manually. This plugin uses the older build method and is not recommended.
\subsection{Apache NetBeans}
\paragraph{}
Clone the repository to a location of your choosing. In NetBeans go to \verb|File| $\longrightarrow$ \verb|Open Project...| and select the cloned repository folder. To access the \verb|LEGv8_Simulator| files in the IDE go to\newline \verb|Graphical-Micro-Architecture-Simulator| $\longrightarrow$ \verb|Modules| and double click on \verb|LEGv8_Simulator|. This will open the module in the project browser. In order to build and package the simulator, right click on \verb|LEGv8_Simulator| $\longrightarrow$ \verb|Run Maven| $\longrightarrow$ \verb|Goals...| and write package into the \verb|Goals| text field and press \verb|OK|. The folder containing the simulator will appear under\newline \verb|LEGv8_Simulator/target/|.

\backmatter

%*******************************************************
% Bibliography
%*******************************************************
\nocite{*}
\bibliography{Bib/Bibliography}
\bibliographystyle{unsrt}

%*******************************************************
% Final dedications and acknowledgements
%*******************************************************
\newpage
% Do not modify this part
\chapter*{}
\thispagestyle{empty}
\vspace*{3cm}
% Do not modify this part

\begin{center}
\hfill Ei fu. Siccome immobile, \\
\hfill Dato il mortal sospiro, \\
\hfill Stette la spoglia immemore \\
\hfill Orba di tanto spiro \\ \medskip
\hfill Alessandro Manzoni -- \emph{Il Cinque Maggio}
\end{center}

\end{document}