\chapter{Building and modernizing the project}\label{chap:chap2}

\epigraph{``Thank goodness we don't have only serious problems, but ridiculous ones as well.''}{\textit{Edsger W. Dijkstra} -- EWD475}

\paragraph{}
In this chapter we will discursively lay out the steps that have been initially taken to get the simulator to build, as intended by the original developers. After this, we will go through the changes made to the project to update its dependencies and integrate it with the Maven \cite{mavenweb} build automation system. Complete step-by-step guides for setting up the original development environment and the newly improved one will be provided in Appendix  \ref{chap:appA} and Appendix \ref{chap:appB} respectively. Although the modernization of the codebase has been the last thing to be achieved in this thesis' work, it is a step that can be considered independent from the development efforts described in the next chapters, and as such it was decided to discuss it in conjunction with the original build system to offer a more consistent exposition.

\section{Reconstructing the original set-up process}
\paragraph{}
As was mentioned in Section \ref{chap:chap1simoverview}, an initial look at the repository's source code and \verb|README| makes evident a lack of documentation over how to set-up the development environment to build and deploy the project. Additionally, exploring the Issues section of the GitHub's repository \cite{legv8simARMrepogit} makes it clear that there have been some failed attempts at getting the codebase to compile, with some users even pointing out bugs in the code but being unable to build the project themselves and produce a fixed version. It can be safely asserted then, that the lack of documentation has been at least partially responsible for the project sitting dormant throughout the years and not receiving needed improvements which would have solved some of the issues discussed in this thesis. Thus, this initial effort of understanding the set-up process for the simulator has been of crucial importance both for this thesis' work and to enable future external contributions.
\paragraph{}
Through an initial phase of trial and error looking at the project's structure, the presence of a \verb|.project| file indicates that the simulator has been at least partially developed using the Eclipse IDE \cite{web:eclipse}, so this will be our starting point to get things back to a working state.
\subsection{Installing the Eclipse IDE and JDK}
\paragraph{}
The Eclipse IDE can be downloaded from its official repository \cite{web:eclipse}. At the time these steps were originally taken, circa the end of 2023, the current version of Eclipse was 2023-09. Due to some breaking changes introduced to the IDE in later updates, it is recommended to download the 2023-09 version of Eclipse for working with this specific project. Once the download is completed, the IDE for Java can be installed normally following the recommended process.
\paragraph{}
After this, it is time to choose an appropriate JDK version. For the time being, JDK version 8 is the safest bet when working with older projects such as this one. One can choose to download any distribution of JDK 8. The choice made for this thesis has been Amazon's Corretto 8 \cite{web:amazoncorretto8}. Once installed or extracted, we can tell Eclipse to use this version of the JDK as default.
\paragraph{}
Once the project has been imported into Eclipse, the IDE will complain about some components missing. These will be our next clues.
\subsection{GWT, AceGWT, and the final set-up}
\paragraph{}
After the project has been imported, Eclipse will automatically recognize the presence of some GWT code and will suggest installing the GWT plug-in \cite{web:eclipsegwtplugin} from the Eclipse Marketplace. Although the automatic procedure presents some problems and should be canceled, we can manually install the plug-in ourselves.
\paragraph{}
After the plug-in installation, the IDE presents us with some configuration errors. Tnd the place where missing dependencies and configuration problems are displayed is the Build Path section of the project's settings. By going there, a few things are made evident: the project is missing a GWT SDK library and it requires an additional project, AceGWT, to work.
\subparagraph{}
We can download GWT from the official website \cite{gwtweb}. We might be tempted to download the latest version, but with some intuition about the project's age and from a cursory glance at the \verb|LEGv8_Simulator.gwt.xml| file, we see that the version needed is 2.7 \cite{gwt2.7web}. Having completed the download, we can tell Eclipse to use this library for the GWT SDK, and, after a short configuration, the dependency will be satisfied.
\subparagraph{}
Similarly, we need to download the AceGWT library version 1.0.0 (currently, the only one available) from the Releases section of its GitHub repository \cite{web:acegwtgit}. After doing so, we can import the sources into the project, which will appear as an additional Java package inside of the project's tree. Again, the dependency will be satisfied and the IDE will drop its error.
\paragraph{}
GWT uses an XML file for configuration. This file requires a custom XML schema to specify its syntax in the form of a DTD file. This file is specified at the beginning of the XML document, and can either be a link to an online resource or the path to a local file on the developer's file system. Since the project uses GWT 2.7, and AceGWT uses GWT 2.8.2, two versions of this schema are required. The schema in both cases is specified as an URL, so we can either enable Eclipse's automatic downloads from the program's settings, or we can download the DTD files manually and point to their path inside the XML files. Both options are valid, although the second one is recommended, since we cannot be sure of the future availability of those online resources. In any case, Eclipse will drop the remaining issues and the project can be finally built.
\subsection{Building and deploying the project}
\paragraph{}
After installing the GWT plug-in, a new button will have appeared on the toolbar of the IDE, bearing the GWT logo. In order to build the project, we can either click that button or right click on the project's root, navigate to the GWT option, and start the compile process. The build process can be customized with differing levels of obfuscation for the final JavaScript code. In our case, the most verbose option is recommended. After compilation has ended, the old version of the simulator inside the \verb|/war| folder will be overwritten with the new one.
\paragraph{}
It is clear from this description of the original set-up, and  even more evident in Appendix \ref{chap:appA}, that this process is quite long, fragile, and dependent on many user interactions and the visual appearance of the Eclipse IDE. In the next sections we will discuss the improvements that have been made to it, following some steps inspired by the newer features of the latest versions of GWT and AceGWT.
\section{Updating GWT, AceGWT, and the JDK}
\paragraph{}
By browsing the changelogs of more recent versions of GWT, and by taking a look at the tutorial page \cite{web:gwtnewtutorial}, we learn that the library has moved from its original build process to Maven. The \verb|README| for AceGWT also mentions a switch to Maven, although only for its source code version and not the prepackaged 1.0.0 release. Because of these changes, and because of the aims of this thesis, integrating the simulator's codebase with Maven became the natural path to take.
\subparagraph{What is Maven?}
Maven \cite{mavenweb} is a build automation system developed by the Apache Software Foundation \cite{web:apachefoundation}. It is a tool to simplify and automate the management of dependencies, configuration, compilation, and deployment of Java-based applications.
\subsubsection{The JDK}
\paragraph{}
We start with GWT. Its latest version, at the time of writing, is version 2.11. As can be seen from its changelog \cite{web:gwt2.11}, this version deprecates the old method for building GWT applications, introduces compatibility with JDK 21, and adds emulation for some JDK 11 features. This allows us to upgrade the JDK all the way to the latest long term version, and unlock some new Java syntax to use in the simulator. We can thus begin with downloading JDK 21 LTS from our preferred vendor and instruct Eclipse to use it.
\subsubsection{AceGWT's GWT}
\paragraph{}
Since AceGWT uses GWT internally and has moved to Maven in its latest version, we can attempt to build an updated version.  It is only necessary to change GWT's XML configuration to point it to the newest version, and then update Maven's \verb|pom.xml| file to use the latest repository of the library. This is necessary not only because of the newer version number, but also because, in the meantime, the project has been abandoned by Google and left in the hands of the community. In fact, both the simulator and AceGWT were developed before this change happened. Because of this, GWT's package and repository have been stripped of the original Google denomination and taken on a new name.
\paragraph{}
Having performed these small changes, and following the new Maven's procedure to build GWT projects, this updated version of AceGWT can be compiled into a working library. This means that, unfortunately, we cannot use AceGWT's original repository anymore and have instead to rely on this custom version. In order to make these steps more reproducible, a more detailed description of this operation can be found in Appendix \ref{chap:appB}.
\paragraph{}
Everything is now ready to integrate Maven into the main project.
\section{Porting the simulator to Maven}
\paragraph{}
Now that all the dependencies have been properly upgraded to their latest version using Maven, integrating them into the simulator becomes much more straightforward. All that is needed is to follow GWT's new tutorial to build a GWT project using Maven, and configure the \verb|pom.xml| files accordingly.
\paragraph{}
The source of the project is separated into two layers. An external one containing the license, contribution guides, \verb|README|, etc., and an internal layer containing the source of the project itself. Since we want to keep the structure as close as the original as possible, we will need two \verb|pom.xml| files. One, the ``father'',  to describe the project externally, and one, the ``child'', to represent the project's code. The father \verb|pom.xml| is configured as a standard Maven project, as the heavy lifting will be done by the other one. The \verb|LEGv8_Simulator| folder will be the root of the actual GWT application. It contains the project's source code under \verb|/src/main| following Maven's convention, a local Maven repository, and the project's \verb|pom.xml| file. 
\subparagraph{Where to put AceGWT?}
The local repository exists to store the AceGWT library. This is because there is currently no AceGWT module in Maven's online central repository, and as such, for the time being,  a local one is needed.  It could be possible, through the use of Maven plug-ins, to treat a GitHub repository as a Maven one. Unfortunately, since the original AceGWT repository does not contain the additions described earlier in the document and is not properly configured, this is currently not an option. It is important to note that using a local Maven repository is heavily discouraged, and one day might even become deprecated.
\subparagraph{How to organize the source code?}
There is a step during the GWT build process that was overlooked when discussing the original method for compiling the project. As we have pointed out in Chapter \ref{chap:chap1}, in addition to the Java code, the project also contains the HTML, CSS, and images necessary to give the application its structure and style. These files, however, do not get automatically generated nor copied to the folder where the simulator is compiled to. Instead, they must be manually added. The reason why this wasn't a problem until now, is because they were present in the \verb|/war| folder from the beginning, so when the build was complete they were already in the correct place. For this reason, the contents of \verb|/main/src| have been divided into the Java source, and the static resources needed for the website, which have been copied from the \verb|/war| folder to an appropriately named one.
\subparagraph{How to configure the pom.xml?}
Everything has been put in place to create Maven's configuration file. We start by specifying its parent is the \verb|pom.xml| file in the root of the project. After that, we use the same syntax as with AceGWT to specify that the project has GWT as a dependency and is a GWT application.  Then, we inform Maven that there is a local repository containing a library in the \verb|.jar| format, and to include it in the build process. We end by including and configuring a plug-in that automatically copies the static web files to the output folder once the build has terminated. This way there is no need to provide them beforehand outside of the sources folder, and it becomes possible to make changes to them that get reflected automatically to the output folder.
\paragraph{}
The configuration is now complete. We will end the chapter with a brief demonstration of the new workflow for setting up and building the simulator. A more detailed account of what has been described in this section is provided in Appendix \ref{chap:appB}.
\section{The updated workflow}
\paragraph{}
By integrating Maven into the project, we not only managed to include updated versions of the various libraries, but we have automatically enhanced the codebase with all the benefits that Maven is able to bring. Among them, we can now set-up, develop, build, and deploy the project entirely through the terminal with just a couple of commands, and without relying on the Eclipse IDE. Not only that, it is now possible to use other Java IDEs that offer Maven compatibility, which opens up the doors to more modern and featureful tools. Tutorials for the most popular Java IDEs, including versions of Eclipse newer than 2023-09, will be provided in Appendix \ref{chap:appB}, as in the next paragraph we will showcase only the headless approach.
\paragraph{}
First, obtain the project and put it in a folder of your choosing and install Maven on your operating system. After moving to the project's folder, run the \verb|mvn package| command inside the \verb|LEGv8_Simulator| subdirectory. The build process will begin, and, once terminated, the folder containing the compiled simulator will appear under \verb|LEGv8_Simulator/target/| instead of \verb|/war|. The specific name of the output folder will depend on how the \verb|pom.xml| file is configured.
\section{Conclusions}
\paragraph{}
In this chapter we have showcased the undocumented, and probably unoptimized, process of setting up and building the original codebase. What followed was a process of modernization and simplification of the entire procedure by integrating Maven in the project and its dependencies. The final result is now a much simpler series of steps that offer not only a high level of automation and customization, but completely decouple the project from its outdated IDE choice and opens it up for seamless collaboration from developers using different tools and settings. The next chapter will discuss the final touches done to bring the simulator to an acceptably working state.