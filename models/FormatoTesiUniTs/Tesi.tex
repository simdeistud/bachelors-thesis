              %******************************************%
              %                                          %
              % Template standard per la Tesi di Laurea  %
              %            di Stefano Bianchi            %
              %                                          %
              %         versione: 27 marzo 2014          %
              %                                          %
              %******************************************%
       
% Se si vuole stampare su entrambi i lati, porre la dicitura 'twoside'
% tra i parametri documentclass
\documentclass[a4paper,11pt,oneside,openany]{book}

\usepackage[english,italian]{babel}

\usepackage[utf8]{inputenc} % Consente l'uso dei caratteri accentati italiani
\usepackage[T1]{fontenc} % Consente l'uso di caratteri speciali appartenenti in genere ad alfabeti non inglesi

\usepackage{fancyhdr} % Consente il rendering della copertina

\usepackage{graphicx} % Consente l'inserimento delle figure nel documento

\usepackage{subfigure} % Consente l'inserimento di sottofigure

\usepackage{tabularx} % Consente l'uso di funzionalità aggiuntive alle tabelle
\usepackage{longtable} % Consente la scrittura delle tabelle su più pagine

\usepackage[titletoc]{appendix} % Consente di porre le appendici in indice

\usepackage{float} % Necessario per inserire immagini flottanti

\usepackage{fancyvrb} % Consente l'utilizzo di finestre Verbatim (testo cche si vuole porre così com'è, senza formattazione alcuna) all'interno di una cornice

\usepackage{amsmath} % Consente l'uso di funzioni aggiuntive per le formule
\numberwithin{equation}{section}

\usepackage{hyperref} % Indice degli argomenti e rimandi cliccabili, per una navigazione più comoda all'interno del file

\usepackage{lipsum} % Consente di inserire testo fittizio all'interno della tesi. Inutile in fase di scrittura della tesi. Eliminare questa dicitura e tutte le diciture \lorem all'interno del testo.

% Per inserire i contenuti, riferirsi ai file relativi e segnati tra parentesi graffe nei parametri \input

%*******************************************************
% Impostazioni della singola pagina e della copertina
%*******************************************************
\input{Settings/Impostazioni}

%*******************************************************
% Copertina
%*******************************************************
% Imposta la copertina
% Non modificare questa parte
\usepackage[some]{background}
\SetBgScale{1}
\SetBgContents{
\includegraphics[scale=1]{img/units_logo.pdf}}
\SetBgColor{gray}
\SetBgAngle{0}
\SetBgOpacity{0.07}
% Non modificare questa parte

% Titolo della tesi e autore
\title{Reverse engineering per la decodifica di filmati di videosorveglianza in formato proprietario}
\author{Nome Cognome}
\begin{document}
\pagenumbering{roman}
\begin{titlepage}
    \begin{center}
    % Insert the background here (front page)
    \BgThispage
    {\LARGE {\bfseries UNIVERSIT\`A DEGLI STUDI DI TRIESTE \\}}
    \vspace{.5cm}
    {\Large {\bfseries Dipartimento di Ingegneria e Architettura \\}}
    \vspace{1cm}
    \includegraphics[width=6cm,height=6cm]{img/units_logo.pdf}\\[1.5cm]

    {\LARGE
        Laurea Magistrale in Ingegneria Informatica \\
    }
    \vspace{1cm}
    {\LARGE 
        {\bfseries Reverse engineering per la decodifica di filmati di videosorveglianza in formato proprietario}
    }
    \vspace{1cm}

    % Ricordarsi di aggiornare la data prima di inviare in stampa il documento :)
    {\large \today \\
    }

    % Tabella per inserire i nomi del laureando, del relatore e dell'eventuale correlatore nel modo migliore all'interno della pagina
    \vfill
    \begin{table}[h]
        {\large
            \begin{tabular}{c c c c r c c | c c l}
                & & & & Laureando & & & & & Relatore \\
                & & & & \bfseries Nome Cognome & & & & & \bfseries Chiar.mo Prof. Sergio Carrato \\ % Modificare il nome del laureando
                & & & & & & & & & \\
                & & & & & & & & & Correlatore \\ % Nel caso in cui non si ha il correlatore,
                & & & & & & & & & \bfseries Tit. Nome Cognome \\ % eliminare queste due righe
            \end{tabular}
        }
    \end{table}
    Anno Accademico 2013/2014
    \end{center}
\end{titlepage}


%*******************************************************
% Dedica
%*******************************************************
% Non modificare questa parte
\phantomsection
\thispagestyle{empty}
\vspace*{3cm}
% Non modificare questa parte

\begin{center}
Così tra questa \\
immensità s'annega il pensier mio: \\
e il naufragar m'è dolce in questo mare. \\
--- Giacomo Leopardi -- da \emph{L'infinito} ---
\end{center}

\medskip

\begin{center}
\hfill \emph{Dedica 1} \\
\hfill \emph{Dedica 2} \\
\hfill \emph{...} \\
~ \\
\hfill \emph{Altre frasi} \\
\hfill \emph{da dedicare} \\
\hfill \emph{alle persone care} \\
\end{center}
	
\frontmatter

%*******************************************************
% Sommario
%*******************************************************
\chapter{Sommario}

\lipsum[1]

%*******************************************************
% Indice
%*******************************************************
\tableofcontents

%*******************************************************
% Introduzione
%*******************************************************
% Non modificare questa parte
\renewcommand{\chaptermark}[1]{\markboth{\MakeUppercase{\ #1}}{}}
% Non modificare questa parte

\chapter{Introduzione}
% Riempire con ciò che si vuole nell'introduzione
\lipsum

%*******************************************************
% Capitoli
%*******************************************************
\mainmatter

% Non modificare questa parte
\renewcommand{\chaptermark}[1]{\markboth{\MakeUppercase{\chaptername\ \thechapter.\ #1}}{}}
% Non modificare questa parte

% Inserire tutti i capitoli necessari. Si possono inserire tutti i capitoli
% all'interno di un singolo file .tex oppure un file .tex per ogni capitolo.

\chapter{Lorem}\label{chap:lorem}

\section{Ipsum}

Esempio di Citazioni o riferimenti~\cite{einstein}\cite{latexcompanion}

\begin{figure}[h!]\label{fig:example}
\centering
\includegraphics[width=.8\textwidth]{img/units_logo.png}
\caption{Esempio di figura}
\end{figure}

\lipsum[3]

\begin{figure}[!ht]
\centering
\subfigure[Sottofigura 1]{\includegraphics[width=.45\textwidth]{img/units_logo.png}}
\subfigure[Sottofigura 2]{\includegraphics[width=.45\textwidth]{img/units_logo.png}}
\subfigure[Sottofigura 3]{\includegraphics[width=.45\textwidth]{img/units_logo.png}}
\subfigure[Sottofigura 4]{\includegraphics[width=.45\textwidth]{img/units_logo.png}}
\caption{Esempio di figura con sottofigure}
\label{fig:subexample}
\end{figure}

\lipsum[4]

\section{Dolor}
\lipsum[3]

\begin{table}
\centering
\begin{tabular}{|c|c|}
\hline
\textbf{Lorem} & \textbf{Ipsum} \\
\hline
\hline
Dolor & Sit \\
\hline
Amet &  \\
\hline
\end{tabular}
\caption{Esempio di tabella}
\label{tab:example}
\end{table}

\lipsum[4-5]
\input{Main/Capitolo2}
\chapter{Lorem Ipsum}\label{chap:lipsum}

\lipsum[1]

\section{Dolor}
\lipsum[2-4]

\subsection{Sit amet}
\lipsum[5-12]
\input{Main/Capitolo4}

%*******************************************************
% Capitolo conclusivo
%*******************************************************
\input{Back/Conclusioni}

%*******************************************************
% Appendice
%*******************************************************
% Non modificare questa parte
\pagestyle{fancy}
\renewcommand{\chaptermark}[1]{\markboth{\MakeUppercase{APPENDICE\ \thechapter.\ #1}}{}}
% Non modificare questa parte

\appendix
\label{appendix}

\input{Back/Appendice1}
\input{Back/Appendice2}
\input{Back/Appendice3}

\backmatter

%*******************************************************
% Bibliografia
%*******************************************************
\nocite{*}
\bibliography{Bib/Bibliografia}
\bibliographystyle{plain}

%*******************************************************
% Dedica finale o ringraziamenti
%*******************************************************
% Non modificare questa parte
\chapter*{}
\thispagestyle{empty}
\vspace*{3cm}
% Non modificare questa parte

\begin{center}
\hfill Ei fu. Siccome immobile, \\
\hfill Dato il mortal sospiro, \\
\hfill Stette la spoglia immemore \\
\hfill Orba di tanto spiro \\ \medskip
\hfill Alessandro Manzoni -- \emph{Il Cinque Maggio}
\end{center}

\end{document}